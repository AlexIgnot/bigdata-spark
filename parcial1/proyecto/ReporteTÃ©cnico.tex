\documentclass{article}

\title{\textbf{Universidad Veracruzana }}
\date{\textbf{Facultad de Negocios y Tecnologías}}

\begin{document}

\maketitle

\textsf{\Large Experiencia Educativa: Paradigmas de Programación. \\}
 
\maketitle
\textsf{\Large Catedratico: Centeno Tellez Adolfo. \\}

\maketitle
\textsf{\Large Alumno: Villagómez Ígnot Angel Alexander. \\}


\maketitle
\textsf{\Large Tema: Reconocimiento de figuras con RNA de Hopfield. \\}


\maketitle
\textsf{\ Grupo: 402 ISW 1° Parcial \\}
\maketitle
\textsf{\ Fecha de Entrega: 17 de Marzo del 2023 \\}

\newpage
\maketitle
\maketitle
%\section{Integrantes}
\textsf{\ \\
\textbf{Introducción:}\\
\\
En este proyecto utilizamos las Redes Neuronales Artificiales de Hopfield (RNA) para el reconocimiento de patrones o figuras en matrices y compartiré mi experiencia al momento de aplicarlas para dicho fin. \\}
\\

\newpage
\maketitle
\textsf{\textbf{Desarrollo:}}
\textbf{Las Redes de Hopfield:}
Empecé eligiendo las figuras a reconocer, estas fueron algunas (de las varias) vocales del alfabeto Hangul (coreano). Usando matrices de 5 * 8 para representarlas, primero usé un excel para intentar simular las casillas y ver si las figuras podrían utilizarse.




\textbf{Conclusión:}
Al empezar el tema con el profesor y ver las fórmulas escritas en las diapositivas pensé que sería un tema imposible de entender, después de las explicaciones del profesor sobre las funciones que representan las fórmulas, para qué sirven y cómo funcionan, llevar el concepto y abstraerlo en MatLab para reconocer las figuras elegidas fue más sencillo de lo que pensé.
Para mi sorpresa, los patrones que representé en las matrices si fueron reconocidos con este algoritmo.


\textbf{Bibliografía:}
\begin{itemize}
    \item P. Gómez Gil. INAOE,(2017).
\end{itemize}

\end{document}